В первых строках нашей программы указаны три директивы:

	\begin{CCode}{main.c}
			#include <stdlib.h>
			#include <fcntl.h>
			#include <errno.h> \end{CCode}

В нашем случаее они нужны для использования open (2), close (2) и perror (3), а также макросов EXIT\_FAILURE И EXIT\_SUCCESS.

Файлы с расширением .h, указанные внутри <{>} у директивы \#include, являюся заголовочными (или подключаемыми) и могут содержать:
	\begin{itemize}
		\item прототипы функций; 
		\item структуры;
		\item союзы (union);
		\item перечисления (enum);
		\item макросы;
		\item объявления типов (typedef);
		\item глобальные переменные.
	\end{itemize}

Сформулируем общее определение для заголовочных файлов:

\begin{defi}{Заголовочный файл}
	файл, содержимое которого при компилляции автоматически добавляется препроцессором в исходный текст программы на место специальной директивы.
\end{defi}

\begin{important}			
	Заголовочные файлы \textbf{не должны} содержать описания функций и системных вызовов (только их прототипы).
\end{important}
